\documentclass[a4paper]{article}

\usepackage[french]{babel}
\usepackage[utf8]{inputenc}
\usepackage{amsmath}
\usepackage{graphicx}
\usepackage[colorinlistoftodos]{todonotes}

\title{Rapport CTR : Kd-Trees}

\author{Gwendal Le Moulec}

\date{\today}

\begin{document}

\section{Présentation des articles}

\subsection{Construction de kd-Trees en O(N log N)}

\textit{Cette sous-partie traite exclusivement de l'article} On building fast kd-Trees for Ray Tracing, and on doing that in O(N log N)\textit{, de Ingo Wald et Vlastimil Havran.}

L'article présente un algorithme permettant de construire des kd-Trees efficaces avec une complexité de $O(N.log(N))$. Les auteurs partent du constat qu'on cherche en général à construires des kd-Trees efficaces pour la recherche dans un espace à $k$ dimensions, sans prendre en compte le temps de construction du kd-Tree. Il est vrai que ceci n'est pas un gros problème pour des petits espaces. Cependant, les espaces deviennent de plus en plus grands et de plus en plus denses, ce qui met eu premier plan la nécessité de disposer d'algorithmes rapides. Or, la plupart des algorithmes utilisés aujourd'hui ont une complexité de $O(Nlog²N)$ ou même de $O(N²)$. Il est dit que la limite inférieure théoriquement atteignable est $O(N.log(N))$.

La proposition d'algorithme est tirée d'un algorithme en $O(N.log²(N))$ ré-adapté. Comme pour tous les algorithmes, de construction de kd-Trees, ce qui est déterminant est la recherche du plan séparateur en chaque noeud. Pour ce faire, il faut trouver un plan aligné sur un des axes de la base du repère qui minimise le côut estimé de traversée des deux voxels. L'approche utilisée pour calculer ce coût est l'heuristique SAH\footnote{Surface Area Heuristique}. Le coût alors calculé est~:

%TODO formule du coût.

Des considérations heuristiques permettent de définir un ensemble de plans candidats, à partir desquels sera choisi le plan minimisant le coût. L'idée de base pour trouver les candidats est de ne sélectionner que les plans qui délimitent les AABB\footnote{Axis Aligned Bounding Box}, car ce sont les endroits où le nombre de triangles change d'un c\^oté et de l'autre du plan, ce qui provoque des sauts dans la fonction de co\^ut. Les minimums sont donc à chercher à ces positions.

Le principe de l'algorithme initial étudié est le suivant, étape par étape~:
\begin{enumerate}
	\item Avant de conbstruire l'arbre, pour chaque dimension de l'espace, on cherche les plans candidats
\end{enumerate}

\section{Some \LaTeX{} Examples}
\label{sec:examples}

\subsection{How to Leave Comments}

Comments can be added to the margins of the document using the \todo{Here's a comment in the margin!} todo command, as shown in the example on the right. You can also add inline comments:

\todo[inline, color=green!40]{This is an inline comment.}

\subsection{How to Include Figures}

First you have to upload the image file (JPEG, PNG or PDF) from your computer to writeLaTeX using the upload link the project menu. Then use the includegraphics command to include it in your document. Use the figure environment and the caption command to add a number and a caption to your figure. See the code for Figure \ref{fig:frog} in this section for an example.

%\begin{figure}
%\centering
%\includegraphics[width=0.3\textwidth]{frog.jpg}
%\caption{\label{fig:frog}This frog was uploaded to writeLaTeX via the project menu.}
%\end{figure}

\subsection{How to Make Tables}

Use the table and tabular commands for basic tables --- see Table~\ref{tab:widgets}, for example.

\begin{table}
\centering
\begin{tabular}{l|r}
Item & Quantity \\\hline
Widgets & 42 \\
Gadgets & 13
\end{tabular}
\caption{\label{tab:widgets}An example table.}
\end{table}

\subsection{How to Write Mathematics}

\LaTeX{} is great at typesetting mathematics. Let $X_1, X_2, \ldots, X_n$ be a sequence of independent and identically distributed random variables with $\text{E}[X_i] = \mu$ and $\text{Var}[X_i] = \sigma^2 < \infty$, and let
$$S_n = \frac{X_1 + X_2 + \cdots + X_n}{n}
      = \frac{1}{n}\sum_{i}^{n} X_i$$
denote their mean. Then as $n$ approaches infinity, the random variables $\sqrt{n}(S_n - \mu)$ converge in distribution to a normal $\mathcal{N}(0, \sigma^2)$.

\subsection{How to Make Sections and Subsections}

Use section and subsection commands to organize your document. \LaTeX{} handles all the formatting and numbering automatically. Use ref and label commands for cross-references.

\subsection{How to Make Lists}

You can make lists with automatic numbering \dots

\begin{enumerate}
\item Like this,
\item and like this.
\end{enumerate}
\dots or bullet points \dots
\begin{itemize}
\item Like this,
\item and like this.
\end{itemize}
\dots or with words and descriptions \dots
\begin{description}
\item[Word] Definition
\item[Concept] Explanation
\item[Idea] Text
\end{description}

We hope you find write\LaTeX\ useful, and please let us know if you have any feedback using the help menu above.

\end{document}