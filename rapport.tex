\documentclass[a4paper]{article}

\usepackage[french]{babel}
\usepackage[utf8]{inputenc}
\usepackage{amsmath}
\usepackage{graphicx}
\usepackage[colorinlistoftodos]{todonotes}

\title{Rapport CTR : Kd-Trees}

\author{Gwendal Le Moulec}

\date{\today}

\begin{document}
\maketitle

\begin{abstract}
Your abstract.
\end{abstract}

\section{Introduction}

Aujourd'hui, la modélisation 3D doit faire face à des scènes de plus en plus complexes. Cette croissance doit être suivie par des algorithmes de plus en plus efficaces pour gérer la position des objets dans l'espace. Cette classe de problèmes utilise généralement une structure d'arbre dénommée \textbf{kd-Tree}\footnote{kd-Tree : représentation d'un espace de k dimensions sous forme d'arbre binaire. La racine représente l'espace entier et chaque fils représente un sous-espace de l'espace père défini par coupure par un plan $P$ perpendiculaire à l'un des axes du repère orthogonal.}. Pour le calcul de la vue en 2D obtenue à partir d'un certain endroit de la scène par exemple, la technique du lancé de rayons utilise cette structure. Ainsi, lorsque l'on cherche les points d'intersection d'une demie-droite avec les objets de la scène, il suffit de parcourir l'arbre en ne considérant que les noeuds sous-espaces pour lesquels il y a effectivement intersection. Un exemple d'espace 2D partitionné et le kd-tree associés sont respectivements représentés en figures \ref{} et \ref{}. Sur l'arbre, les chemins parcourus pour chercher les points d'intersection avec les sous-espaces sont en rouge et les feuilles pour lesquelles il y a effectivement intersection sont marqués d'un I.

%TODO insérer les figures

Cette méthode est bien plus efficace qu'un parcours de la liste des triangles présents dans la scène. En effet, pour un nombre $N$ de triangles, la complexité d'une recherche "naïve" est d'ordre $O(N)$, alors que l'utilisation d'un kd-tree amène une complexité d'ordre $O(log(N))$.

\section{Présentation des articles}

\subsection{Construction de kd-Trees en O(N log N)}

\textit{Cette sous-partie traite exclusivement de l'article} On building fast kd-Trees for Ray Tracing, and on doing that in O(N log N)\textit{, de Ingo Wald et Vlastimil Havran.}

L'article présente un algorithme permettant de construire des kd-Trees efficaces avec une complexité de $O(N.log(N))$.

\subsection{Parallélisation de la construction sur GPU}

\textit{Cette sous-partie traite exclusivement de l'article} Real-Time KD-Tree Construction on Graphics Hardware\textit{, de Kun Zhou, Qiming Hou, Rui Wang et Baining Guo.}

\section{Some \LaTeX{} Examples}
\label{sec:examples}

\subsection{How to Leave Comments}

Comments can be added to the margins of the document using the \todo{Here's a comment in the margin!} todo command, as shown in the example on the right. You can also add inline comments:

\todo[inline, color=green!40]{This is an inline comment.}

\subsection{How to Include Figures}

First you have to upload the image file (JPEG, PNG or PDF) from your computer to writeLaTeX using the upload link the project menu. Then use the includegraphics command to include it in your document. Use the figure environment and the caption command to add a number and a caption to your figure. See the code for Figure \ref{fig:frog} in this section for an example.

%\begin{figure}
%\centering
%\includegraphics[width=0.3\textwidth]{frog.jpg}
%\caption{\label{fig:frog}This frog was uploaded to writeLaTeX via the project menu.}
%\end{figure}

\subsection{How to Make Tables}

Use the table and tabular commands for basic tables --- see Table~\ref{tab:widgets}, for example.

\begin{table}
\centering
\begin{tabular}{l|r}
Item & Quantity \\\hline
Widgets & 42 \\
Gadgets & 13
\end{tabular}
\caption{\label{tab:widgets}An example table.}
\end{table}

\subsection{How to Write Mathematics}

\LaTeX{} is great at typesetting mathematics. Let $X_1, X_2, \ldots, X_n$ be a sequence of independent and identically distributed random variables with $\text{E}[X_i] = \mu$ and $\text{Var}[X_i] = \sigma^2 < \infty$, and let
$$S_n = \frac{X_1 + X_2 + \cdots + X_n}{n}
      = \frac{1}{n}\sum_{i}^{n} X_i$$
denote their mean. Then as $n$ approaches infinity, the random variables $\sqrt{n}(S_n - \mu)$ converge in distribution to a normal $\mathcal{N}(0, \sigma^2)$.

\subsection{How to Make Sections and Subsections}

Use section and subsection commands to organize your document. \LaTeX{} handles all the formatting and numbering automatically. Use ref and label commands for cross-references.

\subsection{How to Make Lists}

You can make lists with automatic numbering \dots

\begin{enumerate}
\item Like this,
\item and like this.
\end{enumerate}
\dots or bullet points \dots
\begin{itemize}
\item Like this,
\item and like this.
\end{itemize}
\dots or with words and descriptions \dots
\begin{description}
\item[Word] Definition
\item[Concept] Explanation
\item[Idea] Text
\end{description}

We hope you find write\LaTeX\ useful, and please let us know if you have any feedback using the help menu above.

\end{document}