\documentclass[a4paper]{article}

\usepackage[french]{babel}
\usepackage[utf8]{inputenc}
\usepackage{amsmath}
\usepackage{graphicx}
\usepackage[colorinlistoftodos]{todonotes}

\title{Rapport CTR : Kd-Trees}

\author{Gwendal Le Moulec}

\date{\today}

\begin{document}

\section{Lectures complémentaires}

Les deux articles se basent sur l'heuristique SAH. Or, celle-ci est fondée sur une approximation de la distribution des rayons lumineux~: ils sont considérés uniformément distribués et forment une ligne infinie, non bloquée par les objets de la scène. (Ingo Wald et al., section 3). Même si la méthode SAH est reconnue d'après les auteurs, il s'avère que cette approximation est éloignée de la réalité, d'après \cite{anti-sah}. En effet, il n'existe qu'un nombre limité de sources de lumières dans une scène, autrement dit les origines des rayons lumineux ne sont pas uniformément distribuées. Cela implique que selon l'angle de vue, un grand nombre de triangles peuvent être invisibles. Dans ce cas, le coût de traversée de ces triangles ne doit pas être considéré.

D'autres hypothèses sont formulées et devraient être vérifiées, essentiellement le fait que le coût de traversée d'un voxel et d'un triangle sont connus.
\\\\
L'algorithme de Wald et al. n'utilise pas de parallélisme. D'après \cite{parallel-cpu}, le parallélisme n'est pas très adapté à la méthode à cause du besoin de trier. Le problème est que la parallélisation du tri nécessite un nombre important de processus pour être efficace, surtout sur un grand nombre d'objets à trier, ce qui est précisément le cas pour les triangles d'un scène classique.
\\\\
L'algorithme de construction d'arbre kd sur GPU a le mérite d'être une méthode jusqu'ici peu concurrencée. Il est cité dans un bon nombre d'articles sans être fondamentalement remis en cause ou déclaré dépassé. Bien entendu, quelques améliorations ont été proposées, comme l'utilisation d'une autre heuristique estimant mieux la distribution des rayons \cite{anti-sah}.

\section{Ouverture}
Comme cela est rappelé par Zhou et al., les kd-trees ne sont pas seulement utilisés pour le tracé de rayons. La recherche des plus proches voisins se sert également de cette structure. Il faut néanmoins adapter l'arbre  car les besoins ne sont pas les mêmes \cite{buffered-kdtree}. Comme ceci est expliqué par Zhou et al., les composants de la scène ne sont plus des triangles mais des points, ce qui simplifie certaines étapes de l'algorithme (ex~: plus besoin de cacluler les AABBs). L'heuristique proposée n'est plus SAH, mais VVH, qui prend en compte le volume d'un nœud.
\\\\
Il existe d'autres moyens qui permettent d'accélérer la recherche de plus proches voisins ou le lancé de rayons, comme les BVH (Bounding Volume Hierarchies). L'optimisation de leur construction sur GPU est aussi sujette aux recherches \cite{bvh}. Ces structures ont l'avantage de simplifier la répartition des triangles qui ne sont classés que dans un nœud (et non deux à la fois pour les triangles chevauchant le plan de séparation). Cela diminue la taille des structures de données et la mémoire requise pour la construction de l'arbre/BVH. Les auteurs de \cite{bvh} estiment ainsi avoir besoin de jusqu'à dix fois moins de mémoire (section 5) pour 1 million de triangles, soit 100MB. Néanmoins, l'étape cruciale de recherche des meilleurs candidats est plus complexe pour les BVH qui cherchent des volumes d'encadrement pour chacun d'entre eux, alors que les arbres kd n'utilisent que le volume encadrant le nœud à diviser et les plans candidats.

\begin{thebibliography}{1}
	\bibitem{anti-sah} LIANG, Xiao, YANG, Hongyu, QIAN, Yinling, et al. A Fast Kd-tree Construction for Ray Tracing based on Efficient Ray Distribution. Journal of Software, 2014, vol. 9, no 3, p. 596-604.
	\bibitem{parallel-cpu} SHEVTSOV, Maxim, SOUPIKOV, Alexei, et KAPUSTIN, Alexander. Highly Parallel Fast KD-tree Construction for Interactive Ray Tracing of Dynamic Scenes. In : Computer Graphics Forum. Blackwell Publishing Ltd, 2007. p. 395-404.
	\bibitem{buffered-kdtree} GIESEKE, Fabian, HEINERMANN, Justin, OANCEA, Cosmin, et al. Buffer kd trees: processing massive nearest neighbor queries on GPUs. In : Proceedings of The 31st International Conference on Machine Learning. 2014. p. 172-180.
	\bibitem{bvh} LAUTERBACH, Christian, GARLAND, Michael, SENGUPTA, Shubhabrata, et al. Fast BVH construction on GPUs. In : Computer Graphics Forum. Blackwell Publishing Ltd, 2009. p. 375-384.
\end{thebibliography}

\section{Some \LaTeX{} Examples}
\label{sec:examples}

\subsection{How to Leave Comments}

Comments can be added to the margins of the document using the \todo{Here's a comment in the margin!} todo command, as shown in the example on the right. You can also add inline comments:

\todo[inline, color=green!40]{This is an inline comment.}

\subsection{How to Include Figures}

First you have to upload the image file (JPEG, PNG or PDF) from your computer to writeLaTeX using the upload link the project menu. Then use the includegraphics command to include it in your document. Use the figure environment and the caption command to add a number and a caption to your figure. See the code for Figure \ref{fig:frog} in this section for an example.

%\begin{figure}
%\centering
%\includegraphics[width=0.3\textwidth]{frog.jpg}
%\caption{\label{fig:frog}This frog was uploaded to writeLaTeX via the project menu.}
%\end{figure}

\subsection{How to Make Tables}

Use the table and tabular commands for basic tables --- see Table~\ref{tab:widgets}, for example.

\begin{table}
\centering
\begin{tabular}{l|r}
Item & Quantity \\\hline
Widgets & 42 \\
Gadgets & 13
\end{tabular}
\caption{\label{tab:widgets}An example table.}
\end{table}

\subsection{How to Write Mathematics}

\LaTeX{} is great at typesetting mathematics. Let $X_1, X_2, \ldots, X_n$ be a sequence of independent and identically distributed random variables with $\text{E}[X_i] = \mu$ and $\text{Var}[X_i] = \sigma^2 < \infty$, and let
$$S_n = \frac{X_1 + X_2 + \cdots + X_n}{n}
      = \frac{1}{n}\sum_{i}^{n} X_i$$
denote their mean. Then as $n$ approaches infinity, the random variables $\sqrt{n}(S_n - \mu)$ converge in distribution to a normal $\mathcal{N}(0, \sigma^2)$.

\subsection{How to Make Sections and Subsections}

Use section and subsection commands to organize your document. \LaTeX{} handles all the formatting and numbering automatically. Use ref and label commands for cross-references.

\subsection{How to Make Lists}

You can make lists with automatic numbering \dots

\begin{enumerate}
\item Like this,
\item and like this.
\end{enumerate}
\dots or bullet points \dots
\begin{itemize}
\item Like this,
\item and like this.
\end{itemize}
\dots or with words and descriptions \dots
\begin{description}
\item[Word] Definition
\item[Concept] Explanation
\item[Idea] Text
\end{description}

We hope you find write\LaTeX\ useful, and please let us know if you have any feedback using the help menu above.

\end{document}